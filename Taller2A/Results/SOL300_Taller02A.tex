\documentclass[10pt]{article}
\usepackage{amsmath,amsfonts,amssymb,amsthm,amscd,enumerate}
\usepackage[utf8x]{inputenc}
\usepackage[spanish,english]{babel}
\usepackage{epsfig}
\usepackage{listings}

\newcommand{\TODO}[1]{\begingroup\color{red}#1\endgroup}

\author{Diego Esteban Quintero Rey}

\begin{document}



\section{Taller 2A: Uso de comandos de shell para procesar archivos}

\section{Ejercicio Final}

Utilizando los comandos anteriores:
1. Cree un nuevo archivo con información de interés para usted. Archíve los resultados  en el directorio Result/. Indique que combinaciones y operador usted ha utilizado
2. Utilize al menos tres combinaciones de comandos para generar nueva información.

\subsection*{Solución}

\subsubsection*{¿Cuáles codones terminan en "G" y se traducen a un aminoácido terminado en "ine"? (13 líneas)}

\begin{lstlisting}[language=sh]
grep "G,.*ine" Data/genetic_code.txt | cut -d ',' f1
\end{lstlisting}

\subsubsection*{¿Cuáles son las secuencias de Drosophila simulans? (178 líneas)}

\begin{lstlisting}{[language=sh]}
grep -A1 "Dro.*lans" Data/mature.fa | grep -v '>' | grep -v '-' 
\end{lstlisting}

\subsubsection*{Liste las especies que tienen secuencias que no contienen "A" (35 líneas)}

\begin{lstlisting}{[language=sh]}
grep -B1 -v "A" Data/mature.fa | grep '>' | cut -d ' ' -f3,4 | sort | uniq
\end{lstlisting}

\begin{itemize}
  \item Obtenemos líneas que no contienen "A" (Todos los encabezados tienen "A". Solo se obtienen secuencias.)
  \item Incluímos la linea inmediatamente anterior a cada línea del paso anterior con "-B1".
  \item Obtenemos los encabezados con '>'.
  \item Obtenemos las columnas 3 y 4 que corresponden al nombre de la especie.
  \item Ordenamos los resultados
  \item Obtenemos una línea por cada especie con "uniq"
\end{itemize}

\end{document}
